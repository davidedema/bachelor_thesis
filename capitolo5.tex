\chapter{Conclusione}
\label{cha:conclusione}
Nel seguente elaborato è stato presentato il mio progetto sia di tirocinio che di tesi triennale. Tutte le fasi dello sviluppo, dalla progettazione alla realizzazione, sono state descritte in modo dettagliato.
In questo capitolo conclusivo verranno presentati i risultati ottenuti e le considerazioni riguardo al progetto sviluppato. Infine verrà dedicata una sezione per le possibili evoluzioni future del progetto.
Iniziamo quindi discutendo dei risultati ottenuti e delle considerazioni.

\section{Risultati ottenuti e considerazioni}
\label{sec:risultati}
Il prodotto finale a cui sono arrivaato è un task planner sviluppato in prolog, il goal principale del planner è quello di risolvere il problema della costruzione di pilastri con dei blocchi di lego.
Questo poi è stato incapsulato in un nodo ROS per rendere il planner utilizzabile in un contesto reale (e simulato). 
Il goal prefissato dal prof. Palopoli, quindi, è stato raggiunto e completato nella sua interezza.

Lo sviluppo di questo progetto mi ha fatto prendere coscienza del mondo della programmazione dichiarativa e della pianificazione automatica. 
Ritengo che Prolog sia uno strumento molto potente che permette di risolvere problemi complessi in modo semplice e veloce.
Prolog è sicuramente uno dei linguaggi che più si adattano a risolvere problemi di questo tipo, le ragioni sono molteplici, qui di seguito elenco quelle che ritengo più importanti:
\begin{itemize}
    \item Paradigma dichiarativo: il programmatore non deve preoccuparsi di come risolvere il problema, ma deve soltanto descriverlo. 
          Questo è particolarmente adatto ai problemi di pianificazione dove è importante descrivere il dominio del problema piuttosto che la soluzione e prolog è perfetto per questo scopo.
    \item Motore di inferenza: Prolog dispone di un motore di inferenza integrato che utilizza un algoritmo di ricerca ad-hoc per raggiungere gli obiettivi specificati dal programma. 
          La potenza sta nell'esplorazione automatica delle soluzioni, questo rende il planner più semplice poichè il ragionamento è delegato al motore di inferenza.
    \item Backtracking: il backtracking consente di esplorare diverse "strade" per raggiungere lo stesso obiettivo. Se una di queste non dovesse funzionare, Prolog "torna indietro" per provarne un'altra fino a trovarne una valida.
    \item Pattern matching: il potente strumento di pattern matching di prolog (unificazione) permette di confrontare stati con obiettivi e condizioni per stabilire che azione compiere.
    \item Logica del prim'ordine: la logica dei predicati di prim'ordine permette a prolog di essere più espressivo e di poter descrivere in modo più semplice e naturale il dominio del problema.
    \item Rappresentazione della conoscenza: prolog permette di rappresentare la conoscenza in modo semplice e naturale, questo è molto importante per la pianificazione automatica poichè è necessario descrivere il dominio del problema.
\end{itemize} 

Inoltre il progetto mi ha permesso di approfondire le mie conoscenze riguardo al mondo della robotica e dell'intelligenza artificiale affermandone anche il mio interesse. In questi mesi di lavoro il prof. Palopoli ha costruito un vero e proprio team composto dal prof. Roveri, il dott. Lamon e il dott. Saccon.
Essendo questo progetto una sfida nuova che non avevo mai affrontato, ho dovuto svolgere molta ricerca e studio per poterlo portare a termine. Loro sono risultati fondamentali per questo scopo siccome erano sempre disponibili al confronto e alla discussione.

I video demo dei risultati ottenuti sono disponibili nella repository del progetto \cite{gitrepo}.
\section{Sviluppi futuri}
\label{sec:sviluppifuturi}
In questa sezione verranno presentati alcuni possibili sviluppi futuri del progetto. Quelli identificati da me sono principalmente 3 ma questo non significa che le possibilità di sviluppo siano limitate a queste. Riporto qui di seguito i possibili sviluppi futuri identificati:
\begin{itemize}
      \item Temporalizzazione delle azioni per una scelta più efficiente del piano da eseguire.
      \item Istanziamento dei fatti tramite rete neurale.
      \item Scenari di paralelizzazione con più agenti (legato molto al primo punto).
\end{itemize}

\subsection*{Temporalizzazione delle azioni}
\label{subsec:temporalizzazione}
In questo progetto le azioni sono sequenziali e atemporali, questo significa che non è possibile eseguire più azioni contemporaneamente e che non è possibile specificare il tempo di esecuzione di un'azione.
Oltre a queste implicazioni la scelta del piano non è ottimizzata quindi non verrà scelto il piano più veloce ma il primo che rispetta i vincoli.

Un lavoro futuro potrebbe essere quindi quello di aggiungere la temporalizzazione delle azioni e i vincoli temporali tra di esse. Così facendo si potrebbe ottimizzare la scelta del piano utilizzando una strategia di minimizzazione del \textit{makespan}\footnote{Tempo che trascorre da inizio a fine piano}.
Si otterrebbe quindi una scelta del piano con criterio e quindi più efficiente e intelligente. 

Un possibile modo per integrare questo aspetto è quello di usare le varie librerie di CLP\footnote{Constraint Logic Programming} presenti in SWI-Prolog. Questa permette di creare dei vincoli tra variabili e di risolverli in modo efficiente.

\subsection*{Utilizzo di una rete neurale per istanziare i fatti}
\label{subsec:neuralnet}
Un altra possibile evoluzione sarebbe legata all'istanziazione dei fatti. Come già presentato nei precedenti capitoli, i fatti \verb+block/13+ al momento sono \textit{grounded}. 
Questo significa che sono sempre veri siccome sono stati definiti a priori. 

Un possibile miglioramento sarebbe quindi di utilizzare un implementazione di prolog neuroprobabilistica, questo renderebbe possibile istanziare i fatti tramite una rete neurale.
L'idea sarebbe quindi di collegare un sensore, ad esempio una telecamera, al mio programma prolog e di utilizzare i dati ricevuti da essa e una rete neurale per istanziare i fatti \verb+block/13+.
Un implementazione di prolog neuroprobabilistica è \textit{DeepProbLog} \cite{MANHAEVE2021103504} ma ne esistono anche altre.

\subsection*{Scenari di paralelizzazione con più agenti}
\label{subsec:parallel}
L'ultimo scenario identificato è quello della possibilità di paralelizzare il piano per la cooperazione multi-agente.
Questo sviluppo è molto legato al primo punto, infatti per poterlo realizzare è necessario che le azioni siano temporali e che quindi sia possibile eseguirle contemporaneamente.
La sfida quindi sarebbe inizialmente di dividere le azioni che ho identificato in sotto azioni e quindi capire quando la possibilità di cooperazione tra agenti è possibile. 
Il dott. Saccon sta già lavorando su questo aspetto e quindi questo potrebbe essere un possibile sviluppo futuro del progetto che sarà applicato realmente nell'immediato futuro.
