\chapter*{Sommario} % senza numerazione
\label{sommario}

\addcontentsline{toc}{chapter}{Sommario} % da aggiungere comunque all'indice
Nel corso di questo elaborato di tesi viene descritto lo sviluppo di un task planner attraverso il linguaggio Prolog.
Questo progetto è stato sviluppato durante il mio periodo di tirocinio interno presso l'università.
Il prof. Palopoli mi ha proposto questo lavoro in quanto interessato a sviluppare un task planner in prolog per poi valutare quali potessero essere le possibili applicazioni.
Ho scelto questo progetto in quanto mi permetteva di approfondire i miei interessi in ambito di intelligenza artificiale e robotica.
Inoltre mi ha consentito di affinare le mie abilità di ricerca e di problem solving proprio perchè era una nuova sfida che fin dal principio non aveva una soluzione unica e ben definita.  

Il compito principale del planner è quello di generare un piano di azioni che permetta ad un agente, robotico o umano, di raggiungere un determinato obiettivo.
In particolare deve risolvere il problema della creazione di un pilastro di blocchi di lego. Ognuno di questi blocchi è caratterizzato da dimensione, forma, posizione nello spazio e attributi di contatto.
Ricavato il piano di azioni, questo deve essere eseguito da un agente, nel mio caso un braccio robotico UR5. 

Il planner è stato sviluppato in Prolog, un linguaggio di programmazione logica dichiarativo, che permette di esprimere il dominio del problema affrontato in modo semplice e naturale.
Questo è possibile perchè è basato sulla logica dei predicati di prim'ordine, consentendo così di esprimere la conoscenza in modo astratto e generale.
Questo dà la facoltà di avere un codice molto compatto e leggibile, ma allo stesso tempo molto potente, flessibile e capace di risolvere problemi complessi.

Il mio lavoro è iniziato quindi studiando il linguaggio e le sue caratteristiche, dando molta importanza allo stato dell'arte. 
Successivamente ho approfondito il problema del task planning e ho cercato di capire come poterlo risolvere. Anche in questo caso lo studio di lavori già esistenti è stato essenziale e mi ha permesso di capire quali fossero le possibili soluzioni.

Dopo questa fase preliminare, ho iniziato a sviluppare il planner. Questa è risultata la parte più impegnativa di tutto il progetto, in quanto non avevo mai utilizzato Prolog prima d'ora e non avevo mai sviluppato un task planner.
La sfida quindi è stata quella di capire come poter risolvere il problema e come poterlo implementare in Prolog. In questa fase è stato fondamentale confrontarsi con il prof. Palopoli e con gli altri supervisori: il prof. Roveri, il dott. Lamon e il dott. Saccon.
Loro sono stati fondamentali per offrire un punto di vista diverso e per aiutarmi a capire come poter sviluppare il planner. 

Una volta sviluppato il planner, ho iniziato a lavorare sul codice che permetteva di interfacciare il planner con il robot. Questo viene fatto attraverso ROS, un framework che permette di sviluppare applicazioni robotiche in modo semplice e modulare.
Il codice Prolog è stato wrappato inizialmente in Python e poi è stato utilizzato ROS per creare la comunicazione.

Creata poi la simulazione è stato testato il tutto e sono state apportate delle modifiche per migliorare il codice e per risolvere alcuni problemi che si sono presentati durante la fase di test. 
Risolti questi ultimi, il caso di studio è stato simulato con Gazebo. 

Il risultato atteso dal prof. Palopoli e da me è stato confermato ed il planner è stato in grado di risolvere il problema. 
Il progetto è stato sviluppato interamente da me, ma è stato fondamentale il supporto di tutto il team introdotto prima. 
Qualsiasi lavoro utilizzato per lo sviluppo del mio progetto e la scrittura della tesi è stato citato ed è quindi consultabile in bibliografia.

