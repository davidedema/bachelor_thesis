\clearpage
\newpage
\chapter{Descrizione del caso di studio}
\label{cha:descrizionecasostudio}
In questo capitolo verrà presentato più in dettaglio il caso di studio. Verrà definito inizialmente il workflow adottato durante il periodo di tirocinio, spiegando le milestone principali raggiunte.
Successivamente verranno illustrate le parti principali del programma Prolog, approfondendo il funzionamento e le sfide incontrate durante la sua creazione.

Lo sviluppo del task planner è stato oggetto del mio periodo di tirocinio, il lavoro è durato complessivamente circa 4 mesi. Durante questo periodo ho dovuto inizialmente studiare il linguaggio partendo dalle sue basi, sviluppare un idea e successivamente implementarla.
In robotica un task planner è un componente software che si occupa di pianificare e coordinare le attività del robot al fine di raggiungere determinati obiettivi.
Il task planner prende in input il goal, analizza il constesto in cui il robot opera e restituisce in output un piano che indica l'ordine delle azioni da eseguire per completare con successo il compito assegnato.
Una volta generato, questo viene eseguito dal sistema di controllo del robot, nel mio caso il \textit{motion planner}. In sostanza, il task planner è responsabile della pianificazione delle attività del robot in modo efficiente e intelligente. È uno dei componenti fondamentali per il controllo e automazione delle applicazioni robotiche complesse.

Descriviamo ora le milestone individuate per lo sviluppo del progetto:
\begin{itemize}
    \item Prendere dimestichezza con Prolog: fase preliminare di studio del linguaggio
    \item Sviluppare il codice in Prolog: fase di sviluppo del codice
    \item Wrappare il codice Prolog con un linguaggio ad alto livello: fase di ricerca e studio della libreria adatta per il wrapping
    \item Creare la simulazione in Gazebo e ROS: fase di sviluppo della simulazione
\end{itemize}
In questo capitolo voglio entrare nel merito delle prime due milestones, la terza e la quarta verranno approfondite nel capitolo \ref{cha:descrizionearchros}.
Queste milestone verranno contestualizzate con una data indicativa sia per l'inizio che per la fine. Per ognuna verrà indicato il lavoro svolto e il suo contributo per la realizzazione del caso di studio.

\section{Prima milestone: Studio di Prolog}
\label{sec:studioprolog}
In questa sezione andremo a vedere in dettaglio la prima milestone, ovvero lo studio del linguaggio Prolog. Questa è stata fondamentale per lo sviluppo del task planner, in quanto mi ha permesso di capire le potenzialità del linguaggio e come sviluppare il progetto.
Prolog ha una curva di apprendimento molto ripida, infatti il tempo che ho dedicato a questa fase è stato all'incirca di un mese e mezzo, da marzo fino a metà aprile.
Durante questo periodo ho studiato lo stato dell'arte del linguaggio, sia in ambito robotica e task planner che non (vedere capitolo \ref{cha:statoarte}) e mi sono esercitato con esso.

Inizialmente sono stato introdotto al linguaggio tramite video su YouTube e articoli online. Elemento fondamentale per
lo studio del linguaggio è stato il libro \textit{Prolog Programming for Artificial Intelligence} \cite{bratko2001prolog} e il materiale fornito da esso.
Questo conteneva delle esercitazioni che modellavano il noto problema del \textit{Blockworld}. In queste lezioni infatti veniva spiegato come costruire una base di conoscenza che contenesse al suo interno
le informazioni necessarie per descrivere lo stato dei blocchi nello spazio tridimensionale. L'esercizio supponeva che ci fosse una telecamera posta sopra il nostro 'mondo' e che fosse ortogonale ad esso, quindi la coordinata z non poteva essere calcolata con la sola osservazione.
I blocchi erano rappresentati con il fatto \verb+see/3+, questo identificava un blocco alla coordinata X e Y specificata.
C'era anche un predicato aggiuntivo che specificava il contatto del blocco, questo era il predicato \verb+on/2+. Questo serviva a specificare se il blocco si trovasse sopra un altro blocco oppure sopra il tavolo.
Un esempio di base di conoscenza utilizzante questi fatti è il seguente:
\begin{lstlisting}
    see(a,2,5).
    see(d,5,5).
    see(e,5,2).

    on(a,b).
    on(b,c).
    on(c,table).
    on(d,table).
    on(e,table).
\end{lstlisting}
Come si può notare da questo esempio, l'ipotetica telecamera individua solamente i blocchi \textit{a}, \textit{d}, \textit{e} siccome  \textit{b} e \textit{c} sono sotto \textit{a}.
Essendo uno spazio tridimensionale doveva essere identificata anche la coordinata z. Questa veniva calcolata tramite un approccio ricorsivo. La ricorsione in questi paradigmi di programmazione è ampiamente usata e questo primo esempio mi è servito per capire come funzionasse.
L'algoritmo per calcolare la coordinata z di un blocco era il seguente:
\begin{lstlisting}
    z(Block,0) :- on(Block, table).
    z(Block,Z) :- on(Block, OtherBlock), z(OtherBlock, Z1), Z is Z1 + 1.
\end{lstlisting}
Questo è un classico esempio di algoritmo ricorsivo in cui si ha un caso base e un caso ricorsivo. Il caso base indica che se il blocco si trova sul tavolo la sua coordinata sarà 0, quello ricorsivo invece itera finchè non trova il blocco che poggia sul tavolo, poi ricorsivamente incrementa di 1 la coordinata z.

In seguito, verso fine marzo inizio aprile, ho provato ad ampliare questo esempio cercando un modo efficiente di rappresentare la realtà. È quindi iniziato il periodo di sviluppo del codice in cui il programma ha iniziato a prendere forma. Tutte le varie versioni del programma sono documentate e reperibili nella repository GitHub del progetto \cite{gitrepo}.
Vediamo ora come è stato sviluppato il codice del planner e le sfide incontrate.
\section{Seconda milestone: Sviluppo del planner}
\label{sec:sviluppoplanner}
In questa sezione approfondiremo le fasi di creazione del planner, il codice del planner e le sfide incontrate durante lo sviluppo.
Il periodo effettivo di creazione del codice del planner è iniziato verso metà aprile ed è finito all'incirca a metà giugno. Ritengo che questo sia stato il compito più impegnativo e allo stesso tempo più interessante del progetto.
Impegnativo perchè, come detto prima, Prolog non è un linguaggio semplice da imparare e creare qualcosa da 0 può risultare ostico; interessante perchè mi ha permesso di capire come funzionano i task planner e quali sono le potenzialità di Prolog.
Andiamo quindi a dividere questa sezione in tre parti, cioè le tre parti che compongono il programma: la rappresentazione dei blocchi, le azioni e il generatore del piano.

\subsection{Rappresentazione dei blocchi}
\label{subsec:rapprblocchi}
Una parte fondamentale del mio progetto è stata la scelta della convenzione per rappresentare i blocchi nella base di conoscenza. Questa, come detto nel capitolo \ref{cha:statoarte}, è composta da fatti e regole. I fatti sono le informazioni che il programma ha a disposizione per ragionare, mentre le regole sono le azioni che il programma può eseguire per modificare lo stato del mondo.
In questa versione, che non utilizza la logica probabilistica, i fatti sono istanziati dall'utente. In questo caso si dice che sono \textit{grounded}, quindi veri al 100\%.
Premesso ciò andiamo a vedere come sono stati rappresentati i blocchi, presentando anche l'evoluzione nelle varie versioni.

Inizialmente la rappresentazione era stata ideata con il seguente fatto:
\begin{verbatim}
    block(ID, X, Y, Z, W, H, D, O, S).
\end{verbatim}
L'ID rappresenta un identificativo univoco del blocco, questo può essere ad esempio b1. La tripla X, Y, Z rappresenta le coordinate nello spazio tridimensionale della posizione del blocco, queste vengono usate per localizzare il blocco nello spazio. La tripla W, H, D rappresenta la dimensione del blocco, rispettivamente la lunghezza, l'altezza e la profondità. Infine il parametro O
rappresenta l'orientamento del blocco rispetto al mondo. Per ragioni di semplicità non abbiamo voluto rappresentarlo in una maniera canonica come gli angoli di eulero oppure i quaternioni ma, invece, con un intero che indica una posizione fissata del blocco. Questo intero ha un range da 1 a 6 e questi numeri indicano: 1 = il blocco è posizionato col la testa\footnote{Con testa mi riferisco alla parte superiore del blocco di lego} rivolta verso l'alto (posizione usuale del blocco per essere impilato), 2 = il blocco è sottosopra quindi la testa poggierà sul tavolo, 3 = il blocco poggia sul lato e la testa è verso l'interno del tavolo, 4 = il blocco poggia sul lato e la testa è verso l'esterno del tavolo, 5 = il blocco poggia sul lato e la testa è verso la parte sinistra del tavolo, 6 = il blocco poggia sul lato e la testa è verso la parte destra del tavolo.
S rappresenta la forma del blocco che nel nostro caso di studio sarà sempre un cubo. Questo parametro quindi non verrà mai utilizzato, è stato solamente aggiunto per la completezza.

Questa rappresentazione, però non era ottimale per il mio scopo. Non teneva conto del fatto che i blocchi possono essere impilati e quindi della relazione di contatto tra i blocchi. Dopo varie riflessioni ho quindi deciso di aggiungere due nuovi attributi al mio fatto \verb+block+:
\begin{verbatim}
    block(ID, X, Y, Z, W, H, D, O, CH, CL, S).
\end{verbatim}
Sono stati aggiunti gli attributi CH e CL. Questi servono a specificare il "contatto alto" e il "contatto basso". Con contatto alto mi riferisco al contatto che il blocco ha con la sua testa, mentre con contatto basso mi riferisco al contatto che il blocco ha con la sua base. Ad esempio se il blocco b1 dovesse essere sopra al blocco b2 il suo attributo CL sarebbe b2 mentre l'attributo CH di b2 sarebbe b1. Questo fornisce già un astrazione abbastanza adeguata per la nostra realtà, ma con l'evolvere del progetto non si è rivelata sufficiente.
Era necessario infatti che venisse in qualche modo rappresentato il concetto di blocco composto da due blocchi. Bisognava quindi trovare un modo di creare e modificare i fatti \textit{on the fly} e di rappresentare la nozione di blocco composto.
Il primo problema è stato risolto con l'aggiunta della proprietà \textit{dynamic} al predicato block. Questo permette di modificare i fatti in runtime, quindi di aggiungere e rimuovere fatti. La rimozione e l'aggiunta di nuovi fatti è possibile tramite i predicati built-in \verb+retract/1+ e \verb+assertz/1+. Il primo serve a rimuovere un fatto dalla base di conoscenza, il secondo invece serve ad aggiungerlo.
Per rendere un predicato dinamico serve specificarlo a inizio file con la seguente direttiva: \verb+:- dynamic block/13+ nel caso del blocco.
Per rappresentare invece il blocco composto mi sono servito di due ulteriori attributi; il predicato \verb+block/13+ è quindi così composto:
\begin{verbatim}
    block(ID, X, Y, Z, W, H, D, O, CH, CL, S, MB, L).
\end{verbatim}
Questi ultimi due attributi sono MB e L, servono rispettivamente a identificare i blocchi di cui è composto il blocco e a indicare se il blocco fa parte di un blocco composto o no. Porto un esempio così da essere più chiaro:
\begin{verbatim}
    block(b1, 1, 0, 0, 1, 2, 1, 1, b2, table, block, [b1], 1).
    block(b2, 1, 0, 2, 1, 2, 1, 1, air, b1, block, [b2], 1).
    block(s1, 1, 0, 0, 1, 4, 1, 1, air, table, block, [b1, b2], 0).
\end{verbatim}
Il blocco b1 si trova in coordinate 1, 0, 0, ha dimensioni 1, 2, 1 e sopra di lui si trova b2 e sotto il tavolo, è composto da se stesso e fa parte di un blocco composto.
Il blocco b2 si trova in coordinate 1, 0, 2, ha dimensioni 1, 2, 1 e sopra di lui si trova l'aria e sotto b1, è composto da se stesso e fa parte di un blocco composto.
Infine, il blocco composto s1 si trova in coordinate 1, 0, 0, ha dimensioni 1, 4, 1 e sopra di lui si trova l'aria e sotto il tavolo, è composto da b1 e b2 e non fa parte di un blocco composto.

Questa è la rappresentazione che, ad oggi, ritengo più efficace per rappresentare questa astrazione. Essendoci dei possibili scenari futuri che potrebbero richiedere una rappresentazione diversa, questa potrebbe cambiare in futuro.
\subsection{Azioni del planner}
\label{subsec:azioniplanner}
Ora andremo a vedere il set di azioni che il planner può eseguire e come queste sono state implementate in Prolog. Ho individuato 3 tipi diversi di azioni:
\textit{move\_block}, \textit{rotate\_block} e \textit{link}. Verra riportata la loro implementazione e di seguito una spiegazione del loro funzionamento.
Queste azioni vanno intese come astrazioni ad alto livello di azioni che il robot può compiere. Il compito del task planner quindi non è quello di descrivere nei minimi dettagli il movimento che dovrà compiere il braccio, compito questo del motion planner.
Il suo scopo è di fornire un piano ad alto livello che il robot deve seguire per raggiungere lo stato finale desiderato.

Iniziamo con la \verb+rotate_block/5+; questa azione serve a portare un blocco dal suo orientamento attuale a quello desiderato, ovviamente per i nostri scopi l'orientamento desiderato sarà sempre 1.
L'implementazione che ho scelto per questa azione è la seguente:
\begin{lstlisting}[numbers=left]
rotate_block(Block, X, Y, Z, NO) :-
    block(Block, X, Y, Z, W, H, D, O, TL, TH, S, MB, L),
    (list_length(MB, N), N > 1 -> rotate_list(MB, NO); NO = NO),
    L is 0,
    add_action(rotate(Block, X, Y, Z, NO)),
    retract(block(Block, X, Y, Z, W, H, D, O, TL, TH, S, MB, L)),
    assertz(block(Block, X, Y, Z, W, H, D, NO, TL, TH, S, MB, L)).
\end{lstlisting}
Come si può vedere, la testa della nostra clausola di Horn prende in input 5 parametri: il blocco da ruotare, le sue coordinate e il nuovo orientamento. I primi 4 parametri vengono utilizzati nella riga successiva per cercare di unificare con il predicato \verb+block/13+ e ottenere così le informazioni del blocco.
Avendo unificato il blocco possiamo quindi accedere a tutte le sue proprietà tra cui l'orientamento attuale e se è un blocco composto. Quest'ultimo ad esempio viene usato nell'istruzione successiva. Se stiamo operando un blocco composto da più blocchi allora dobbiamo ruotare anche questi ultimi. Per farlo utilizzo il predicato \verb+rotate_list/2+ che prende in input la lista dei blocchi che compongono il blocco composto e il nuovo orientamento. Questo predicato si occupa di ruotare tutti i blocchi della lista.
\begin{lstlisting}[numbers=left]
rotate_list([], NO).

rotate_list([H|T], NO) :-
    rotate_compose(H, _, _, _, NO),
    rotate_list(T, NO).

rotate_compose(Block, X, Y, Z, NO) :-
    block(Block, X, Y, Z, W, H, D, O, TL, TH, S, MB, L),
    (list_length(MB, N), N > 1 -> rotate_list(MB, NO); NO = NO),
    retract(block(Block, X, Y, Z, W, H, D, O, TL, TH, S, MB, L)),
    assertz(block(Block, X, Y, Z, W, H, D, NO, TL, TH, S, MB, L)).
\end{lstlisting}
Con questa serie di predicati riusciamo appunto a ruotare tutti i blocchi che lo compongono. Iteriamo inizialmente tutti gli elementi nella lista e poi li passiamo uno ad uno al predicato \verb+rotate_compose/5+. Questo è identico al predicato principale, l'unica differenza è che non controlla se un blocco fa parte di un blocco composto oppure no (L=1).

Tornando al predicato principale, successivamente verrà svolto il controllo per verificare che il blocco non faccia parte di un blocco composto. Svolto quest'ultimo controllo possiamo aggiungere l'azione eseguita al piano (verrà spiegata meglio nella sezione \ref{subsec:generatorepiano}) e infine aggiornare il predicato \verb+block/13+ con il nuovo orientamento.
Per svolgere quest'ultima azione utilizzeremo i predicati built-in \verb+retract/1+ e \verb+assertz/1+ che ci permettono di rimuovere un predicato e aggiungerne uno nuovo.

La seconda azione presa in considerazione è la \verb+move_block/7+; questa azione serve a spostare un blocco da una posizione ad un'altra. L'implementazione che ho scelto per questa azione è la seguente:
\begin{lstlisting}[numbers=left]
move_block(Block, X, Y, Z, NX, NY, NZ) :-
    block(Block, X, Y, Z, W, H, D, O, TL, TH, S, MB, L),
    (list_length(MB, N), N > 1 -> move_list(MB, NX, NY, NZ); NX = NX, NY = NY, NZ = NZ),
    L is 0,
    add_action(move(Block, X, Y, Z, NX, NY, NZ)),
    retract(block(Block, X, Y, Z, W, H, D, O, TL, TH, S, MB, L)),
    assertz(block(Block, NX, NY, NZ, W, H, D, O, TL, TH, S, MB, L)).
\end{lstlisting}
Questo predicato è molto simile se non uguale al precedente, cambia solo il fatto che non si deve ruotare il blocco e che si devono aggiornare le coordinate del blocco. La logica quindi è uguale come la sua implementazione, le uniche differenze sono che nel piano verrà aggiunta l'azione \verb+move/7+ e che il predicato \verb+block/13+ verrà aggiornato con le nuove coordinate. Anche in questo caso c'è il controllo sul blocco composto, quindi ci sarà anche l'implementazione di \verb+move_list/4+ che è molto simile a \verb+rotate_list/2+.

L'ultima azione presa in considerazione per il nostro caso di studio è la \verb+link/3+. Questa essenzialmente serve a verificare se i blocchi sono impilati, se li si "incolla" creando il blocco composto. La sua implementazione è la seguente:
\begin{lstlisting}[numbers=left]
link(B1, B2, R) :-
    block(B1, X1, Y1, Z1, W1, H1, D1, O1, TL1, TH1, S1, MB1, L1),
    block(B2, X2, Y2, Z2, W2, H2, D2, O2, TL2, TH2, S2, MB2, L2),
    count(s, N),
    all_diff([B1, B2]),
    L1 = 0,
    L2 = 0,
    X1 = X2,
    Y1 = Y2,
    Z1 is Z2 - H2,
    W1 = W2,
    D1 = D2,
    HighP is H1 + H2,
    add_action(link(B1, B2)),
    retract(block(B1, X1, Y1, Z1, W1, H1, D1, O1, TL1, TH1, S1, MB1, L1)),
    retract(block(B2, X2, Y2, Z2, W2, H2, D2, O2, TL2, TH2, S2, MB2, L2)),
    assertz(block(B1, X1, Y1, Z1, W1, H1, D1, O1, B2, TH1, S1, MB1, 1)),
    assertz(block(B2, X2, Y2, Z2, W2, H2, D2, O2, TL2, B1, S2, MB2, 1)),
    string_concat('s', N, PIL),
    retract(count(s, N)),
    N1 is N+1,
    assertz(count(s, N1)),
    atom_string(STACK,PIL),
    R = STACK,
    assertz(block(STACK, X2, Y2, Z2, W1, HighP, D1, 1, TL2, TH1, block, [B1,B2],0)).
\end{lstlisting}
Iniziamo definendo in input i due blocchi che vogliamo "incollare"; R è un parametro in cui verrà salvato l'ID del blocco composto creato.
Le due istruzioni seguenti servono a recuperare le informazioni dei due blocchi che vogliamo utilizzare cercando di unificarli con i predicati \verb+block/13+.
L'istruzione a linea 5 serve a controllare che i due blocchi non siano uguali, questo perchè non ha senso creare un blocco composto con un solo blocco. Verifichiamo poi un insieme di precondizioni tra cui la verifica che i due blocchi non siano già parte di un blocco composto, che siano allineati e che siano della stessa dimensione.
Successivamente calcoliamo l'altezza del blocco composto come somma delle altezze dei due blocchi. Aggiorniamo la base di conoscenza e creiamo aggiungiamo anche il blocco composto che avrà come ID la stringa "s" seguita da un numero che viene incrementato ad ogni blocco composto creato.

Queste sono le tre azioni base che il planner deve eseguire per poter risolvere il problema. Ora vediamo come vengono utilizzate per generare il piano che poi verrà eseguito dal nodo di motion planning.
\subsection{Generatore del piano}
\label{subsec:generatorepiano}
La generazione del piano è il cuore del planner, è qui che vengono utilizzate le azioni base per generare il piano che verrà eseguito dal nodo di motion planning. Il task planner è composto, per il nostro scopo, principalmente da un predicato: \verb+pillar/7+.
Questo è strutturato nel seguente modo:
\begin{lstlisting}[numbers=left]
pillar(X, Y, Z, High, Width, Depth, Actions) :-
    find_blocks(Blocks),
    valid_blocks(Blocks, High, ValidBlocks, Width, Depth),
    nth0(0, ValidBlocks, B1),
    nth0(1, ValidBlocks, B2),
    stack(B1, B2, X, Y, Z, R),
    select(BL1, ValidBlocks, ValidBlocks1),
    select(BL2, ValidBlocks1, ValidBlocks2),
    stackRec(ValidBlocks2, R, X, Y, Z),
    plan(Actions).
\end{lstlisting}
I primi 3 argomenti sono le coordinate dove verrà costruito il pilastro, i tre seguenti sono le dimensioni del pilastro. L'ultimo argomento è l'output del planner cioè una lista di azioni che verranno eseguite dal nodo di motion planning.
Il predicato inizia quindi salvando nella lista \textit{Blocks} tutti i blocchi utilizzabili, quindi con L=0, presenti nella base di conoscenza.
Successivamente, tramite un algoritmo di esplorazione ricorsiva, trova la combinazione di blocchi adatta per soddisfare l'altezza richiesta dall'utente. Il codice Prolog è il seguente:
\begin{lstlisting}[numbers=left]
valid_blocks(Blocks, DesiredHeight, ResultBlocks, DesiredWidth, DesiredDepth) :-
    length(Blocks, N),               
    between(1, N, NumBlocks),        
    length(ResultBlocks, NumBlocks),  
    select_blocks(Blocks, ResultBlocks, DesiredHeight, DesiredWidth, DesiredDepth). 


select_blocks(_, [], 0.0, DesiredWidth, DesiredDepth). 
select_blocks(Blocks, [Block|Remaining], DesiredHeight, DesiredWidth, DesiredDepth) :-
    seleziona(Block, Blocks, RemaningBlocks), 
    block(Block, _, _, _, DesiredWidth, Height, DesiredDepth, _, _, _, _, _, _), 
    DesiredHeight >= Height,                
    RemainingHeight is DesiredHeight - Height,
    select_blocks(RemaningBlocks, Remaining, RemainingHeight, DesiredWidth, DesiredDepth). 


seleziona(X, [X|Resto], Resto).
seleziona(X, [Y|Resto], [Y|Resto1]) :-
    seleziona(X, Resto, Resto1).
\end{lstlisting}
A prima vista può sembrare complesso ma cercherò di spiegarlo in maniera semplice ed efficace. L'entrypoint di questa "procedura" è \verb+valid_blocks/5+, il compito di questo predicato è di istanziare una lista di lunghezza compresa tra 1 e n, dove n è il numero di blocchi disponibili. Inizialmente la lista verrà istanziata a 1, nel caso l'iterazione fallisca questa verrà incrementata di 1 fino ad arrivare alla capienza massima cioè n.
Successivamente viene chiamato il predicato \verb+select_blocks/5+; questo restituirà la lista contenente i blocchi che soddisfano i requisiti di altezza, larghezza e profondità richiesti dall'utente. Inizialmente quindi il programma Prolog cercherà di trovare una soluzione con un solo blocco, se non la troverà passerà a due blocchi e così via fino ad arrivare ad una soluzione oppure al fallimento del predicato.

Trovata la lista di blocchi che soddisfa i requisiti, il programma prosegue con la creazione del pilastro. Questo viene creato tramite i predicati \verb+stack/6+ e \verb+stackRec/5+, l'ultimo è la versione ricorsiva in cui si passa la lista dei blocchi da utilizzare e l'ID del blocco composto di "base". Questi due predicati utilizzano le azioni spiegate prima per creare il pilastro. L'ultima istruzione, \verb+plan/1+, copierà le azioni create durante la generazione del pilastro nella lista Actions che verrà restituita in output dal predicato \verb+pillar/7+.

Questa è quindi la soluzione al caso di studio che ho pensato e implementato. Ovviamente è migliorabile e ci sono sicuramente altri modi per risolvere questo problema, ma questo è il risultato che ho ottenuto e che ho deciso di presentare in questa tesi.

