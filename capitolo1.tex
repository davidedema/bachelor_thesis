\chapter{Introduzione}
\label{cha:intro}

% - INTRODUZIONE AL TEMA
% - MOTIVI DEL CASO DI STUDIO
% - OBIETTIVI DEL LAVORO
% - STRUTTURA 

In questo primo capitolo viene introdotto il caso di studio, partendo dalla definizione del problema, definendo poi gli obiettivi desiderati
e concludendo con una spiegazione della struttura della tesi.

% INTRODUZIONE AL TEMA
Per la mia tesi triennale mi sono voluto concentrare sul campo dell'intelligenza artificiale e la robotica. Il caso di studio identificato
dal prof. Luigi Palopoli è l'utilizzo di Prolog, un linguaggio logico e dichiarativo, per l'implementazione di un task planner ad alto livello.
Questo deve essere in grado di coordinare un manipolatore robotico, nel nostro caso un UR5 della Universal Robots, per l'operazione di manipolazione di blocchi.
Il planner tramite delle primitive come \textit{ruota} e \textit{muovi} deve essere in grado di pilotare il braccio robotico per la
creazione di un pilastro. Esso è stato concepito ad alto livello per permettere la compatibilità con più tipologie di robot e la flessibilità
delle applicazioni. Nel caso studiato verrà utilizzato per la creazione, appunto, di un pilastro, ma questa astrazione ad
"alto livello" permette di essere usato per più compiti: ad esempio per la creazione di una costruzione come una casa o qualsiasi struttura
che si può creare con dei blocchi di Lego. Queste primitive verrando inviate al nodo controllore del movimento che, tramite algoritmi di cinematica e
interpolazione dei punti, troverà le configurazioni dei giunti adatte al movimento richiesto. 

Successivamente si è voluto implementare il tutto in un
ambiente simulato, così da avere un riscontro anche visivo del lavoro svolto e capire le possibili applicazioni in un contesto reale.
La simulazione è stata svolta con ROS e Gazebo, due strumenti open source ampiamente utilizzati nel mondo della robotica. Il primo ci permette di
comunicare al robot tramite una serie di topic e servizi, mentre il secondo ci permette di simulare l'ambiente in cui il robot si trova.

% MOTIVI DEL CASO DI STUDIO
La scelta di questo caso di studio è nata principalmente dal mio interesse verso la robotica e le applicazioni dell'intelligenza artificiale in vari contesti.
Il caso studiato è stato un approccio nuovo che non avevo mai considerato e fin da subito mi è interessato. L'utilizzo di costrutti logici per rappresentare
la conoscienza in un robot è un approccio molto interessante che può essere applicato in molti altri contesti.
Un altro motivo che mi ha portato a scegliere questo caso di studio è la sua possibilità di espansione. Due possibili miglioramenti potrebbero essere
la temporizzazione delle azioni e la successiva ottimizzazione del tempo
di esecuzione (\textit{makespan}) finale e l'utilizzo del machine learning per istanziare i fatti iniziali.
Sapere che questo lavoro possa essere portato avanti e migliorato è sicuramente un altro motivo principale della scelta di questo caso di studio.
Mi sono quindi rivolto al prof. Palopoli e lui mi ha proposto di sviluppare questo progetto.
La scelta di utilizzare Prolog è nata dal fatto che è un ottimo strumento per modellare la conoscenza in un agente.
Esso permette di creare delle "basi di conoscenza", contenenti sia fatti che regole, su cui successivamente fare inferenza.
L'esecuzione di un programma prolog è comparabile alla dimostrazione di un teorema mediante la regola di inferenza della soluzione.
Tutte queste premesse, e l'apmpio utilizzo di prolog nel mondo dell'intelligenza artificiale, ci hanno incuriosito e portato allo sviluppo del progetto.

\section{Definizione del problema}
\label{sec:defprob}
Il caso di studio presentato non è nuovo nel mondo dell'intelligenza artificiale. L'utilizzo di linguaggi logici per applicazioni di planning è, ed era, uno
degli approcci preferiti dalla comunità per risolvere problemi in questo dominio. È difficile trovare delle applicazioni nella vita reale che utilizzano solamente Prolog, questo il 
più delle volte viene usato insieme ad altri strumenti o linguaggi di programmazione tipo Java. 

Inizialmente la sfida è stata di prendere domestichezza con il linguaggio di programmazione.
Essendo un paradigma che non avevo mai utilizzato e, abituato ai linguaggi imperativi come Python, Java, ho provato a usare un approcio simile
in Prolog, fallendo. Questo perchè, al contrario di come siamo abituati, Prolog fa un uso massiccio della ricorsione e il concetto di variabile è differente da come siamo abituati. 
Presa domestichezza la sfida diventò quella di definire un modo efficace per rappresentare l'ambiente studiato.
Dovevo riuscire a trovare un astrazione della realtà tale che, utilizzando una stringa, riuscisse a rappresentare tutte le informazioni necessarie sulle proprietà
attuali del blocco di lego. Successivamente questa si è trasformata nel identificare le azioni base che il nostro planner doveva inviare al robot.
Tutto ciò doveva essere svolto mantenendo consistenza tra il mondo modellato e la base di conoscenza, quindi evitare che un blocco che si è spostato
nel mondo reale rimanga nella sua posizione precedente nella base di conoscenza. 

\section{Obiettivi desiderati}
\label{sec:obiettdes}
In sede di sviluppo del progetto abbiamo voluto specializzare il planner nella costruzione di pilastri di altezza scelta dall'utente. Questi sono 
composti dai \textit{megablocks}, dei blocchi di lego di dimensione maggiorata, per facilitare la presa al nostro manipolatore. Nella base di conoscenza questi saranno rapressentati come dei fatti \textit{ground}, ciò significa che il nostro 
interprete Prolog li assumerà per veri a priori. Il predicato che svolgerà la creazione del pilastro dovrà restituire tutte le possibili soluzioni che
soddisfano il nostro problema. Successivamente una serie di azioni saranno calcolate per la costruzione del pilastro e queste verranno inviate al robot. Esso è un UR5
del produttore Universal Robots, un manipolatore a 6 gradi di libertà. Alla sua estremità è presente un gripper a due dita parallele della Soft Robotics che permette la presa dei blocchi.
Tramite il piano generato, il robot dovrà costruire il pilastro richiesto con i blocchi messi a disposizione. Riceverà quindi una serie di azioni ad alto
livello e convertirà queste in movimenti veri e propri del braccio tramite il nodo di motion planning.

\section{Struttura della tesi}
\label{sec:struttura}
Questa tesi inizierà presentando lo stato dell'arte delle applicazioni in Prolog, sia l'uso come puro linguaggio logico che nei contesti di
intelligenza artificiale e robotica. Vedremo l'evoluzione del linguaggio negli anni e la sua applicazione in ambiti sia di ricerca che di industria.
Successivamente andrò a spiegare cosa è ROS e gli strumenti che ho utilizzato per simulare il robot. In questa sezione descriverò
l'architettura e i nodi creati per il progetto, soffermandomi anche su concetti di cinematica e di motion planning.
In seguito ci sarà un capitolo dedicato più dettagliatamente al caso di studio in cui il programma Prolog sarà spiegato più
dettagliatamente e infine un capitolo dedicato alle conclusioni e ai lavori futuri.

Iniziamo ora introducendo appunto lo stato dell'arte di prolog e le pubblicazioni notabili, andando a dare un approfondimento anche al mondo dell'industria.

